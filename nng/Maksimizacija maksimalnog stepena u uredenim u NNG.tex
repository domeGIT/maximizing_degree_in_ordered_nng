%%%%%%%%%%%%%%%%%%%%%%%%%%%%%%%%%%%%%%%%%%%%%%%%%%%%%%%%%%%%%%%%%%%%%%%
%
%   Presentation of Beamer UNL Theme
%   Beamer Presentation by Chris Bourke
%
%%%%%%%%%%%%%%%%%%%%%%%%%%%%%%%%%%%%%%%%%%%%%%%%%%%%%%%%%%%%%%%%%%%%%%%

\documentclass[11pt]{beamer}

\usetheme[hideothersubsections]{UNLTheme}

\usepackage[serbian]{babel}
\usepackage{graphicx} % Required for including images
\usepackage{caption}  % Optional, but useful for more options
\usepackage{array} % u preambuli
\usepackage{booktabs}
\usepackage{amsmath}
\DeclareMathOperator{\diam}{diam}
\usepackage{colortbl}
\usepackage{tabularx}
\usepackage{multirow}
\usepackage{microtype}
\renewcommand{\figurename}{Slika} % Changes "Figure" to "Slika"
\setbeamertemplate{theorems}[numbered]
\newenvironment{teorema}
  {\begin{block}{Teorema}}
  {\end{block}}


\title{Maksimizacija maksimalnog stepena u ure\dj enim grafovima najbli\v{z}ih suseda}
%\subtitle{Pregled rada: \textit{Maximizing the maximum degree in ordered nearest neighbor graphs}}
\author{Dimitrije Ra\dj enovi\'{c}} %
\institute{{Matemati\v{c}ki fakultet, Univerzitet u Beogradu}}
\date{januar 2026.}

\begin{document}

%{% open a Local TeX Group
%\setbeamertemplate{sidebar}{}
\begin{frame}
      \titlepage
      \begin{center}
   % \href{mailto:cbourke@cse.unl.edu}{\color{blue}%{\texttt{mr...@matf.rs}}}
       \end{center}
\end{frame}
%}% end Local TeX Group


\section{Op\v{s}te o radu}



%\section{Op\v{s}te o radu}   


\begin{frame}{Autori i afilijacije}
\vspace{-0.1truecm}
\begin{itemize}
 \setlength{\itemsep}{0pt}
  \setlength{\parskip}{0pt}
    \item  
    {\fontsize{10.5}{11}\selectfont \emph{P\'{e}ter \'Agoston} – { \hskip-0.07truecm Alfr\'{e}d  R\'{e}nyi  Institute  of  Mathematics, Budapest, Hungary}.}
    \item {\fontsize{10.5}{11}\selectfont \textls[-20]{\emph{Adrian Dumitrescu}  –   {Algoresearch \hskip-0.05truecm L.L.C., \hskip-0.06truecm Milwaukee, \hskip-0.07truecm WI, \hskip-0.08truecm USA}.}}    
    %i Alfr\'{e}d R\'{e}nyi Institute of Mathematics.
    \item {\fontsize{10.5}{11}\selectfont \emph{Arsenii Sagdeev} –  {Karlsruhe  Institute  of Technology, Karlsruhe, Germany}.}
    \item {\fontsize{10.5}{11}\selectfont \emph{Karamjeet Singh} – {Indraprastha \hskip0.04truecm Institute of  \hskip0.04truecm Information Technology, Delhi, India}.}
   \item   {\fontsize{10.5}{11}\selectfont \textls[-20]{\emph{Ji Zeng} –  {University  of California  San  Diego, La  Jolla,  CA,}}}\\
   {\fontsize{10.5}{10.7}\selectfont \textls[-40]{USA \hskip-0.03truecm \& \hskip-0.04truecm Alfr\'{e}d \hskip-0.03truecm  R\'{e}nyi  \hskip-0.03truecm Institute \hskip-0.02truecm  of \hskip-0.02truecm Mathematics, \hskip-0.03truecm Budapest, \hskip-0.03truecm Hungary.}}  
    %i Alfr\'{e}d    R\'{e}nyi Institute of Mathematics. 
\end{itemize}
%\vspace{-0.2truecm}
\footnotesize Saradnja je zapo\v{c}ela  tokom \textit{Focused Week on Geometric Spanners} (23–29. oktobar 2023, Budimpe\v{s}ta, Ma\dj arska).
\end{frame}

 
\begin{frame}
    \frametitle{Detalji objavljivanja}
\begin{itemize}
%\setlength{\itemsep}{1pt}
 % \setlength{\parskip}{1pt}
 \item \textbf{Naslov rada:} \textit{Maximizing the maximum degree in \hskip 1truecm ordered nearest neighbor graphs}.
  \item \textbf{\v{C}asopis:} \textit{Computational Geometry: Theory and Applications}.
\item \textbf{Tom:} 132 (2026). 
\item \textbf{Broj \v{c}lanka:} 102229.
\item \textbf{Dostupnost online:} od 18. septembra 2025. 
\item \textbf{DOI:} 10.1016/j.comgeo.2025.102229.
 \item Ranija \hskip0.2truecm verzija \hskip0.2truecm  ovog \hskip0.2truecm rada objavljena je u \textit{Zborniku radova \textls[-30]{konferencije CALDAM 2025} (13-15. \hskip-0.03truecm februar 2025,} Koimbator, Indija).  %\textit{$11^{st}$ Conference on Algorithms and Discrete Applied Mathematics}, (13-15. februar 2025, Koimbator, Indija).
 \end{itemize}
\end{frame}

\section{O problemu}
\begin{frame}
    \frametitle{Graf najbli\v{z}ih suseda}

   \textls[-20]{Za dati skup \hskip-0.02truecm ta\v{c}aka \hskip-0.02truecm u \hskip-0.02truecm metri\v{c}kom prostoru, \hskip-0.01truecm \textbf{graf \hskip-0.01truecm najbli\v{z}ih suseda} \\
(engl.~\emph{Nearest \hskip0.02truecm Neighbor \hskip0.02truecm  Graph}, NNG) \hskip0.02truecm je \hskip0.02truecm usmereni graf u kome \\
svaki \hskip-0.05truecm \v{c}vor \hskip-0.05truecm ima \hskip-0.05truecm ta\v{c}no \hskip-0.05truecm jednu \hskip-0.05truecm \textbf{izlaznu ivicu} \hskip-0.04truecm ka \hskip-0.04truecm svom \hskip-0.04truecm najbli\v{z}em \hskip-0.04truecm susedu.}
\vspace{-0.3truecm}
\begin{itemize}
\setlength{\itemsep}{0pt}
  \setlength{\parskip}{0pt}
    \item \textbf{Neure\dj eni NNG}: svaka ta\v{c}ka bira najbli\v{z}u me\dj u \textit{svim} ostalim ta\v{c}kama (Slika 1a).
    \item \textls[-20]{\textbf{Ure\dj eni NNG}:   ta\v{c}ke se pojavljuju \textit{jedna po jedna};  svaka nova ta\v{c}ka bira najbli\v{z}u me\dj u \emph{ve\'{c} uvedenim} ta\v{c}kama (Slika 1b).}
\end{itemize}

\vspace{-0.5truecm}
\begin{figure} [ht!]
\centering
                % Adjust width to fit the column width
                               \includegraphics[width=0.65\textwidth]{Slika1-new.png} 
                               \vspace{-0.35truecm}
              
               \footnotesize 
             \centering Slika 1: a) Neure\dj eni i b) ure\dj eni graf najbli\v{z}ih suseda  nad istim \\ 
            \centering skupom od \v{s}est ta\v{c}aka (\'Agoston et al., 2026).
            \end{figure}
\end{frame}

%\section{O problemu}
\begin{frame}{Problem i njegov zna\v{c}aj}

\begin{itemize}
\item U  neure\dj enom  NNG-u,  stepen  svakog \v{c}vora ograni\v{c}en je
 konstantom koja zavisi samo od dimenzije prostora $\mathbb{R}^d$.
\item \textls[-30]{U ure\dj enom NNG-u,   \textbf{ulazni  stepen}  (broj ta\v{c}aka koje biraju datu \hskip-0.03truecm ta\v{c}ku \hskip-0.03truecm kao \hskip-0.03truecm najbli\v{z}eg \hskip-0.03truecm prethodnika) \hskip-0.03truecm zavisi \hskip-0.03truecm od  \hskip-0.03truecm \textbf{poretka} \hskip-0.05truecm ta\v{c}aka.}

\item \textbf{Cilj rada} \textls[-20]{je da se prona\dj e  \textbf{poredak ta\v{c}aka} koji \hskip0.07truecm \textcolor{red}{maksimizuje}}
\textcolor{red}{maksimalni ulazni stepen},  \emph{tj. \hskip-0.02truecm poredak \hskip-0.02truecm u \hskip-0.02truecm kome \hskip-0.02truecm postoji ta\v{c}ka 
\textls[-30]{koja postaje \hskip-0.02truecm najbli\v{z}i prethodnik \hskip-0.03truecm \v{s}to \hskip-0.03truecm ve\'{c}em \hskip-0.02truecm broju \hskip-0.02truecm narednih ta\v{c}aka.}} 
\item \textls[-40]{Ovo predstavlja \textbf{dualni pristup} klasi\v{c}nom problemu minimizacije maksimalnog stepena.}

\item Re\v{s}avanje ovog problema daje \textbf{zna\v{c}ajan} \textbf{doprinos} teoriji ure\dj enih NNG-ova i ra\v{c}unarskoj geometriji. 
%\item  \textls[-22] {Re\v{s}avanje \hskip-0.03truecm  ovog  \hskip-0.04truecm problema  \hskip-0.04truecm \textbf{zna\v{c}ajano} \hskip-0.05truecm doprinosi  \hskip-0.03truecm analizi \hskip-0.05truecm dinami\v{c}kih geometrijskih struktura, teoriji ure\dj enih geometrijskih grafova i osnovnim pitanjima ra\v{c}unarske geometrije.}

  
\end{itemize}
 \end{frame}
 \section{Rezultati}
   

 

%\textbf{Tabela 1:} Konstruktivne donje granice za maksimalni ulazni stepen u tri okruženja.
%\vskip-0.2cm
%\begin{tabular}{|c|c|}
%\hline
%\cellcolor{blue!15}\textbf{} &\cellcolor{blue!15} \textbf{Donja granica za}\\
%\cellcolor{blue!15}\textbf{Okru\v{z}enje} &\cellcolor{blue!15} \textbf{maksimalni ulazni stepen} \\
%\hline 
%Ta\v{c}ke na pravoj &  $\lceil \log_2 n \rceil$   \\
%\hline
%Ta\v{c}ke u $\mathbb{R}^d$ & $\geq \dfrac{\log_2 n}{4d}$ \\
%\hline
%Op\v{s}ti metri\v{c}ki prostor & $\Omega\!\left(\sqrt{\dfrac{\log n}{\log \log n}}\right)$\\
%\hline
%\end{tabular}








%\textbf{Tabela 1:} Konstruktivne donje granice za maksimalni ulazni stepen u tri okruženja.
%\vskip-0.2cm
%\begin{tabular}{|c|c|}
%\hline
%\cellcolor{blue!15}\textbf{} &\cellcolor{blue!15} \textbf{Donja granica za}\\
%\cellcolor{blue!15}\textbf{Okru\v{z}enje} &\cellcolor{blue!15} \textbf{maksimalni ulazni stepen} \\
%\hline 
%Ta\v{c}ke na pravoj &  $\lceil \log_2 n \rceil$   \\
%\hline
%Ta\v{c}ke u $\mathbb{R}^d$ & $\geq \dfrac{\log_2 n}{4d}$ \\
%\hline
%Op\v{s}ti metri\v{c}ki prostor & $\Omega\!\left(\sqrt{\dfrac{\log n}{\log \log n}}\right)$\\
%\hline
%\end{tabular}

\begin{frame}  
    \frametitle{Rezultat za ta\v{c}ke na pravoj}
    \framesubtitle{}
    \vspace{-0.1truecm}
  \begin{block}{Teorema 1}
  \small
 Za svaki skup od $n$ \hskip0.05truecm ta\v{c}aka na pravoj, \hskip0.05truecm postoji poredak takav da  \textls[-22]{odgovaraju\'{c}i ure\dj eni graf najbli\v{z}ih suseda ima maksimalni ulazni stepen najmanje $\lceil \log n \rceil$. Sa druge strane, postoji skup od $n$ ta\v{c}aka na pravoj takav da za svaki poredak, ulazni stepen svake ta\v{c}ke je najvi\v{s}e $\lceil \log n \rceil$.}
\end{block}
  
\vspace{-0.2truecm}
\textit{Klju\v{c}ne ideje   dokaza}:
    \vspace{-0.4truecm}
    \small
\begin{itemize}
\setlength{\itemsep}{2pt}
  \setlength{\parskip}{2pt}

\item\textls[-20]{\textbf{Gornja granica -}  
\hskip-0.05truecm Skup $P_{k+1}\! =\! P_k \!\cup (3^k\! +\! P_k)$ ima dve polovine razdvojene vi\v{s}e nego \v{s}to je njihov dijametar, pa svaka ta\v{c}ka prima najvi\v{s}e jednu ivicu iz druge polovine. Indukcijom sledi da maksimalni ulazni stepen ne prelazi  $\lceil \log n \rceil$.}
\vspace{-0.05truecm}
 \item  \textls[-20]{\textbf{Donja granica -} \hskip-0.05truecm Konstrui\v{s}e se  algoritam  koji rekurzivno 
  postavlja centar ve\'{c}eg podskupa na po\v{c}etak, pa zatim krajnju ta\v{c}ku druge strane koja se povezuje s njim. Svaki nivo rekurzije dodaje jednu ivicu tom centru, pa je maksimalni ulazni stepen  
 $ \ge\lceil \log n \rceil$.}
\end{itemize}
\end{frame}



\begin{frame}  
    \frametitle{Rezultat za ta\v{c}ke u $\mathbb{R}^d$}
    \framesubtitle{}
    \vspace{-0.2truecm}
     \begin{block}{Teorema 2}
     \small
Za svaki skup od $n$ ta\v{c}aka u $\mathbb{R}^d$, postoji poredak takav da odgovaraju\'{c}i ure\dj eni graf najbli\v{z}ih suseda ima maksimalni ulazni stepen bar $\frac{\log n}{4d}$.
\end{block}
 \vspace{-0.1truecm}
\textit{Klju\v{c}ne ideje  dokaza}:
    \vspace{-0.4truecm}
    \small


\begin{itemize}
\setlength{\itemsep}{0pt}
  \setlength{\parskip}{0pt}
    \item  \textls[-60]{Izra\v{c}unati dijametarski par   $ab$    sa   $|ab|\!\! =\!\! 1$.
Neka  je $A\!\! =\!\! \{p\! \in\! P \!:\! |pa|\! \leq \!|pb|\}$ i $B\! =\! \{p\! \in\! P\! : |pb| \leq |pa|\}$. Bez gubitka op\v{s}tosti, $|A|\!\ge \!|B|$, stoga $|A|\! \geq \! n/2$.}

    \item \textls[-30] {Prema Korolaru 1,  $A$ se podeli na najvi\v{s}e $16^d/2$ podskupova pre\v{c}nika manjeg od $1/2$.  Jedan od njih,             $C \!\subseteq \!A$, sadr\v{z}i bar $n/16^d$ ta\v{c}aka.}
\item \textls[-30] {Pore\dj ati ta\v{c}ke: prvo ta\v{c}ka iz $C$ koja ima maksimalni ulazni stepen u rekurzivnom poretku nad $C$, zatim $b$, pa sve ostale ta\v{c}ke iz $C$, i na kraju ta\v{c}ke iz $P \!\setminus (C\! \cup \!\{b\})$.} 
\item Tu ta\v{c}ku iz $C$ defini\v{s}emo kao centar skupa $P$.

\vspace{0.4truecm}
\footnotesize \textls[-30]{\textbf{Korolar 1:} Neka je $P$ kona\v{c}an skup ta\v{c}aka u $\mathbb{R}^d$ takav da  $\mathrm{diam}(P) \leq 1$. Tada se $P$ mo\v{z}e podeliti na najvi\v{s}e $16^d / 2$ podskupova \v{c}iji je dijametar manji od $1/2$.}
\end{itemize}
 
 \end{frame}
 
% \vspace{-0.4truecm}
%\footnotesize $^*$\textbf{Korolar 1:} Neka je $P$ konačan skup tačaka u $\mathbb{R}^d$ takav da je $\mathrm{diam}(P) \leq 1$. Tada se skup $P$ može podeliti na najviše $16^{d/2}$ podskupova čiji je prečnik manji od $1/2$ ( Ágoston et al., %2026).
 

 
\begin{frame}  
    \frametitle{Rezultat za proizvoljni metri\v{c}ki prostor}
    \framesubtitle{}
      \vspace{-0.1truecm}
   \begin{block}{Teorema 3}
   \small
   Za svaki metri\v{c}ki prostor sa $n$ elemenata,  postoji poredak takav da odgovaraju\'{c}i ure\dj eni graf najbli\v{z}ih suseda ima maksimalni ulazni stepen 
$\Omega\!\left(\sqrt{\frac{\log n}{\log \log n}}\right)$.
    \end{block}
\vspace{-0.2truecm}
\textit{Klju\v{c}ne ideje dokaza}:
    \vspace{-0.2truecm}
    \small
 

 \textls[-20]{Primenom Teoreme~5 za $k = c'\sqrt{\frac{\log n}{\log \log n}}$ (za dovoljno malo $c' > 0$), obezbe\dj uje se postojanje monohromatske specijalne strukture i samim tim postojanje poretka takvog da odgovaraju\'{c}i ure\dj{}eni NNG ima maksimalni ulazni stepen bar $k - 1$, \v{c}ime se dokazuje Teorema~3.}
  \vspace{0.1truecm}

\footnotesize \textls[-20] {\textbf{Teorema 5:} Neka je $K_n^{(3)}$ potpun 3-uniformni hipergraf nad ure\dj enim skupom \v{c}vorova $[n]$, \v{c}ije su ivice obojene crvenom, zelenom ili plavom bojom. Ako $K_n^{(3)}$ ne sadr\v{z}i ni crvenu \emph{clique}  $K_k^{(3)}$, ni zelenu  \emph{forward star} $S_k^{(3)}$, ni plavu \emph{forward star} $S_k^{(3)}$, tada va\v{z}i $n \leq \exp(O(k^2 \log k))$.}

 


\end{frame}

\begin{frame}{Pregled  rezultata}
% Zatim u tabular okruženju:
\vspace{-0.3truecm}
\begin{center}
\renewcommand{\arraystretch}{1.2} % opcionalno: malo više prostora oko teksta
\begin{tabular}{|c|c|}
\hline
%\cellcolor{blue!15}\textbf{} &\cellcolor{blue!15} \textbf{\large{Donja granica za}}\\
\cellcolor{blue!15}\textbf{\large{Okru\v{z}enje}} &\cellcolor{blue!15} \textbf{\large{Maksimalni ulazni stepen}} \\
%\cellcolor{blue!15}\textbf{} &\cellcolor{blue!15} \textbf{} \\
\hline
Prava & $\lceil \log n \rceil$ \hskip1.2truecm (Teorema 1) \\
\hline
 $\mathbb{R}^d$ & $\displaystyle \geq \frac{\log  n}{4d}$ \hskip1.2truecm (Teorema 2)\\
\hline
Proizovljan metri\v{c}ki prostor & $\displaystyle \Omega\!\left( \sqrt{ \frac{\log n}{\log \log n} } \right)$ (Teorema 3) \\
\hline
\end{tabular}
\end{center}
\vspace{-0.3truecm}
\begin{itemize}
 \setlength{\itemsep}{1pt}
  \setlength{\parskip}{0pt}
    \item Za ta\v{c}ke na pravoj,  rezultat je \textbf{optimalan}.
    \item Za ta\v{c}ke u  $\mathbb{R}^d$,  rezultat je \textbf{najbolji mogu\'{c}i} do na faktor  $1/(4d)$.
    \item U proizvoljnim metri\v{c}kim prostorima,   ostaje \textbf{otvoreno pitanje} \textls[-22]{da li se mo\v{z}e posti\'{c}i ve\'{c}i maksimalni ulazni stepen.} 
%\item U svim slučajevima postoji \textbf{rekurzivni algoritam} za konstrukciju redosleda.
\end{itemize}
\end{frame}


 
\section{Povezani \hskip-0.05truecm radovi}
\begin{frame}{Povezanost sa relevantnim radovima}


\begin{itemize}
\vspace{-0.3truecm}
\setlength{\itemsep}{0pt}
  \setlength{\parskip}{0pt}
   \item {\fontsize{10}{10.5}\selectfont\textls[-20]{\emph{Agarwal,  Eppstein  \&  Matou\v{s}ek (1992)}:  ure\dj eni NNG u dinami\v{c}kim algoritmima bez analize ekstrema.  → Rad \emph{\'Agosoton et al. \hskip-0.07truecm (2026)} donosi \textbf{prvu sistematsku analizu} najgoreg slu\v{c}aja.}}

   \item  {\fontsize{10}{10.5}\selectfont\textls[-30]{\emph{Eppstein, \hskip-0.07truecm Paterson \hskip-0.05truecm \& \hskip-0.05 truecm Yao \hskip-0.07truecm (1997)}: ograni\v{c}enje stepena u neure\dj enom NNG-u. → Rad \emph{\'Agosoton et al. \hskip-0.07truecm (2026)} \hskip-0.03truecm postavlja \textbf{novu problematiku}: maksimalni ulazni stepen u ure\dj enom NNG-u.}}

    \item {\fontsize{10}{10.5}\selectfont\textls[-20]{\emph{Bose, Gudmundsson \& Morin (2004)}:   uvo\dj enje  ure\dj enih  $\theta$-grafova.       → Rad \emph{\'Agosoton et al. \hskip-0.07truecm (2026)}  pru\v{z}a \textbf{nov pristup} osnovnom slu\v{c}aju ($k=1$),  u kontekstu maksimalnog ulaznog stepena.}}



   
    \item {\fontsize{10}{10.5}\selectfont\textls[-1]{\emph{He \hskip0.05truecm and \hskip0.05truecm Fox (2021)}: Ramsey‑teorijski \hskip0.05truecm rezultat \hskip0.05truecm za \hskip0.05truecm 3‑uniformne hipergrafove. 
   →   Rad \emph{\'Agosoton et al. (2026)} ostvaruje \textbf{prvu primenu} te metode u ra\v{c}unarskoj geometriji.}}


\end{itemize}
\end{frame}

\section{Primene}
\begin{frame}{Potencijalne primene}
%Rezultati ovog rada se mogu primeniti za:
\vspace{-0.2truecm}
\begin{itemize}
          \item \textls[-20]{\textbf{Dinami\v{c}ke geometrijske strukture:} maksimalni ulazni stepen mo\v{z}e rasti logaritamski s brojem ta\v{c}aka, \v{s}to naru\v{s}ava lokalnu ravnote\v{z}u.}
          
          \item \textls[-30]{\textbf{Kontrolisani redosled i rebalansiranje:} kako bi se o\v{c}uvali balansirani stepeni u dinami\v{c}kim strukturama, potrebno je   kontrolisati \hskip-0.03truecm redosled \hskip-0.03truecm umetanja \hskip-0.02truecm ta\v{c}aka, ili primeniti odgovaraju\'{c}i mehanizam za rebalansiranje.}
    
    \item \textls[-20]{\textbf{Pohlepni algoritmi} (engl. \emph{greedy algorithms})\textbf{:} poredak ta\v{c}aka kriti\v{c}no odre\dj uje strukturu grafa,  \v{c}ak i u najjednostavnijem modelu.}
    
    \item \textls[-30]{\textbf{Ure\dj eni Yao/Theta grafovi:} NNG odgovara slu\v{c}aju $k\! = \!1$; temelj za njihovu generalizaciju na $k\! \geq  2$.}
    
    

 

\end{itemize}
\end{frame}
%\section{Dalji pravci}


\section{Zaklju\v{c}ak}
\begin{frame}{Zaključak}

\begin{itemize}
\setlength{\itemsep}{1pt}
  \setlength{\parskip}{1pt}
\item \textbf{Prvi sistematski rezultat} o maksimalnom ulaznom stepenu u ure\dj enom NNG-u.
  
  \item \textbf{Dualni pristup:} umesto minimizacije, maksimizacija maksimalnog ulaznog stepena.
  
   \item \textls[-30]{\textbf{Dokazane su}  vrednosti maksimalnog ulaznog stepena:}
\begin{itemize}
\setlength{\itemsep}{0pt}
  \setlength{\parskip}{0pt}
    \item na pravoj: $\lceil \log n \rceil$  - optimalno;
    \item  u $\mathbb{R}^d$: $\frac{\log n}{4d}$   - optimalno do  na faktor 1/(4d);
    \item  u  metri\v{c}kim prostorima: $\Omega\!\left(\sqrt{\frac{\log n}{\log \log n}}\right)$    - optimalnost je otvoren problem.
\end{itemize}

\item \textls[-48] {Osim konkretnih rezultata, \textbf{značaj} rada ogleda se u metodološkom doprinosu kombinacijom diskretne geometrije i Ramseyjeve teorije - otvarajući put budućim istraživanjima.}
  

%\item \textbf{Prva sistematska analiza} najgoreg slu\v{c}aja za maksimalni ulazni stepen u ure\dj enim NNG-ovima.
    
% Postavlja \hskip-0.05truecm temelje  za dalja istra\v{z}ivanja u teoriji grafova i ra\v{c}unarskoj geometriji. 


    
\end{itemize}

\end{frame}

\begin{frame}{Dalji pravci istra\v{z}ivanja}

\begin{itemize}
  \item \textbf{Pobolj\v{s}anje rezultata u proizvoljnim metri\v{c}kim prostorima:}  Da li za svaki metri\v{c}ki prostor sa $n$ ta\v{c}aka postoji poredak ta\v{c}aka takav da je maksimalni ulazni stepen u odgovaraju\'{c}em ure\dj enom NNG-u bar $\lceil \log n \rceil$?
  
  \item \textbf{Problem 1 (iz  rada):} Za metri\v{c}ki prostor $V$ sa $n$ elemenata i ta\v{c}ku $v \in V$, neka je $d(v)$ maksimalni ulazni stepen ta\v{c}ke $v$ u ure\dj enom NNG-u, posmatran preko svih $n!$ mogu\'{c}ih poredaka ta\v{c}aka. Mo\v{z}e li suma $\sum_{v} 2^{-d(v)}$ biti ve\'{c}a od $1$?\\
% Ako ne postoji takav prostor, tada postoji tačka $v$ sa 
%$d(v) \geq \lceil \log n \rceil$, što bi značajno poboljšalo %Teoremu~3.
  \item \textbf{Generalizacija na ure\dj ene Yao/Theta grafove za $k \geq 2$:} najavljena za naredni rad \'Agoston i saradnika.

 
\end{itemize}
\end{frame}
\section{Reference}
\begin{frame}
\frametitle{Reference}
\footnotesize{
\begin{thebibliography}{5} % Beamer does not support BibTeX so references must be inserted manually as below
\bibitem[\'Agoston et al., 2026]{p1}  P. \'Agoston, A. Dumitrescu, A. Sagdeev, K. Singh, and J. Zeng, “Maximizing the maximum degree in
ordered nearest neighbor graphs”,   
{\em Computational Geometry: Theory and
Applications}, {\bf 132},  102229 (2026). 

\bibitem[Agarwal, Eppstein \& Matou\v{s}ek, 1992]{p4} P. Agarwal, D. Eppstein, and J. Matou\v{s}ek, “Dynamic 
half-space reporting, geometric optimization, and minimum spanning trees”, in: \emph{Proc. of the 33rd Annual Symposium on Foundations of Computer Science},   80--90 (1992).

\bibitem[Eppstein, Paterson \& Yao, 1997]{p2} D. Eppstein, M.S. Paterson, and F.F. Yao, “On nearest-neighbor graphs", \emph{Discrete Comput. Geom.}, \textbf{17},  263--282 (1997). 
\bibitem[Bose, Gudmundsson \& Morin, 2004]{p3} P. Bose, J. Gudmundsson, and P. Morin, “Ordered theta graphs”, \emph{Comput. Geom.}, \textbf{28},   11--18 (2004).

\bibitem[He and Fox, 2021]{p5} X. He and J. Fox, “Independent sets in hypergraphs with a forbidden link”, \emph{Proc. Lond. Math. Soc.}, \textbf{123},  384--409 (2021).
\end{thebibliography}
}
\end{frame}
%------------------------------------------------
\end{document}
 \section{Reference}
\begin{frame}{Reference}

\end{frame}
\end{document}
\begin{frame}[fragile]
    \frametitle{How To Use The Theme}
    \framesubtitle{Theme Options}
        
    There are also several options that you can pass:
    \begin{itemize}
      \item \texttt{hideothersubsections} -- This hides subsections in the 
            sidebar \emph{other than} the subsections of the current section.
      \item \texttt{hideallsubsections} -- This option doesn't print \emph{any}
            subsections in the sidebar.
      \item \texttt{width} -- sets the width of the sidebar, default is 
      	    \verb"2.5\baselineskip"
      \item \texttt{height} -- sets the height of the header, default is 
      	    \verb"2.5\baselineskip"
      \item \texttt{left} -- sets the sidebar to the left of the slide (default).
      \item \texttt{right} -- sets the sidebar to the right of the slide.
    \end{itemize}
    
\end{frame}

\begin{frame}
    \frametitle{How To Get The Theme}
    \framesubtitle{}
    
    The theme is available from my home page,
    \textcolor{blue}{\url{http://www.cse.unl.edu/~cbourke/}}
    

    You need \texttt{You need beamerthemeUNLTheme.sty} and \texttt{UNL.pdf}
    (the UNL logo).
    
    Place them in the working directory or add them to \texttt{beamer/themes/theme}
    or somewhere in your \LaTeX\ path and you're good to go!
    
\end{frame}
    
\end{document}
